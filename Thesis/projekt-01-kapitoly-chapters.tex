% Autor: Peter Rúček

\chapter{Úvod}

Keď sa zrodí nápad na knihu, je spočiatku jedno, o aký príbeh ide. Skvelý nápad je skvelý nápad. Ale definovanie typu príbehu, pochopenie základných tém a poznanie, kam príbeh komerčne patrí, jasne pomáha určiť štruktúru aj smer.

Pokiaľ ide o žánre a typy zápletiek, počet rôznych druhov sa pohybuje od 1 po čokoľvek. Žánre sa formujú podľa konvencií, ktoré sa časom menia, keďže kultúry vymýšľajú nové žánre a prestávajú používať staré. Diela často zapadajú do viacerých žánrov prostredníctvom vypožičiavania a rekombinácie týchto konvencií. No napriek tomu sa žánre stali najpopulárnejším spôsobom ako klasifikovať umenie.

Naproti tomu typy zápletiek sú viac-menej nemenné, a predsa skoro vôbec nepoužívané. Pre dejové línie typu \textit{morálne dilema} alebo \textit{zneužitie sily} bolo určite napísaných, a určite ešte aj bude, veľa kníh. Definovanie však týchto základných kľúčových dejových línií je problematické. Na toto téma bolo napísaných viacero kníh a článkov, no myšlienky autorov sú veľmi odlišné a nevedia sa zhodnúť ani v tom koľko by ich malo asi byť. Niektorý tvrdia, že ich je 20, iný ich rátajú v tisícoch. 

Táto práca sa zameriava na hľadanie týchto kľúčových dejových línií v zhrnutiach dejov a užívateľských recenziách, ktoré sa získavajú zo sociálnych sietí. Systém je navrhnutý tak, že dokáže pracovať s akýmkoľvek dejom a na základe strojového učenia systém predpovedá o aký typ deja by sa podľa jeho názoru mohlo jednať.

Ciele tejto práce by sa dali využiť v sociálnych sieťach týkajúcich sa nielen kníh, kde by diela mohli byť klasifikované podľa týchto kľúčových dejových línií. Slúžili by pri vyhľadávaní, keď si užívateľ nevie spomenúť na názov knihy alebo by sa mohli takto hľadať podobnosti v dielach a navrhovať tak obdobné pri prezeraní konkrétnej knihy.

Práca je členená do šiestich kapitol. V kapitole \ref{teoria} je uvedený teoretický pohľad, potrebný pre pochopenie problematiky. Zaoberá sa popisom deja, prehľadávaním webu, klasifikačnými metódami a následne uvedením doterajších a aj súčasných prístupov strojového učenia v spracovaní prirodzeného jazyka. Návrhom výsledného systému analýz sa venuje kapitola \ref{navrh}. Nasleduje kapitola \ref{implementacia}, v ktorej je popísaná implementácia tohoto systému. Kapitola \ref{experimenty} popisuje jednotlivé experimenty nad dátami. Záverečná \ref{zaver}. kapitola zhodnocuje ciele systému a dosiahnuté výsledky.

\chapter{Rozbor riešenej problematiky }
\label{teoria}

\todo{Po dopisani teorie dopis toto}

\section{Príbeh}
Základom pre pochopenie kľúčových dejových línií a ich analýzu je pochopenie čo je to príbeh, z čoho sa skladá a ako to súvisí s dejom. Nasledujúce podkapitoly vychádzajú \linebreak
z \cite{Rosaria:2004} a \cite{Roberts:1987}.

\subsection*{Dej}

Dej je základ pre príbeh, založený na protichodných ľudských motiváciách, s činmi vyplývajúcimi z uveriteľnej a realistickej ľudskej reakcie. To znamená, že konflikt je základná časť, ktorú je potrebné vytvoriť, aby sa vytvoril súbor udalostí pri formovaní príbehu. Konflikt určí ďalšiu akciu alebo situáciu. Bude to určujúci faktor pre vytvorenie hlavnej štruktúry príbehu.
 
Dejom príbehu je teda nastolenie konfliktu a dôsledky, variácie a vývoj, ktoré z neho pramenia. Ďalej ním myslíme myšlienku, ktorá určuje, ako bude príbeh plynúť. Teda bude spájať jednu akciu s druhou, aby bol príbeh dobre organizovaný. V dobre vykreslenom príbehu nie je nič irelevantné; všetko spolu súvisí. V príbehu je čas dôležitý nielen preto, že jedna vec sa deje za druhou, ale preto, že jedna vec sa deje kvôli druhej.

Tak ako dej plynie, možno väčšinu dejov zaradiť do tejto tradičnej dejovej štruktúry, viď obrázok \ref{climax} :
\begin{itemize}
	\item \textbf{Expozícia (Úvod):} Expozícia je rozloženie a uvedenie základných prvkov v príbehu: hlavné postavy, ich pozadie, ich charakteristiky, ciele, obmedzenia a potenciál. Predstavuje všetko, čo bude v príbehu dôležité.
    \item \textbf{Kolízia (Zápletka):} Kolízia označuje začiatok veľkého konfliktu v príbehu. Účastníkmi sú protagonisti a antagonisti, spolu s akýmikoľvek myšlienkami alebo hodnotami, ktoré predstavujú, akými sú dobro a zlo, individualizmus a kolektivizmus, láska a nenávisť, inteligencia a hlúposť, vedomosti a nevedomosť, sloboda a otroctvo, túžba a odpor a podobne.
    \item \textbf{Kríza (Vyvrcholenie, Climax):} Kríza je bod zlomu, oddeľujúci medzi tým, čo bolo predtým, a tým, čo príde potom. Naplno sa v ňom prejaví konflikt a následné napätie. Ďalším spôsobom, ako myslieť na vyvrcholenie, je definovať ho ako bod v príbehu, v ktorom sa všetka ostatná akcia stáva nevyhnutnou.
    \item \textbf{Peripetia (Dejový obrat):} Peripetiou rozumieme nečakaný dejový zvrat, nepredpokladanú zmenu situácie. Dej sa zrýchľuje a smeruje ku koncu.
    \item \textbf{Katastrofa (Rozuzlenie):} Rozuzlenie je súbor akcií, ktorými sa príbeh končí. Hlavné akcie sú dokončené a posledná akcia podčiarkuje tón konečnosti.
\end{itemize}
\begin{figure}[hbt]
	\centering
	\includegraphics[width=0.8\textwidth]{obrazky-figures/climax.png}
	\caption{Štruktúra deja. Prebrané z \cite{ClimaxImage}.}
	\label{climax}
\end{figure}



\subsection*{Postavy}
Postavy sú osoby prezentované v diele, ktoré čitateľ interpretuje ako osoby obdarené morálnymi a dispozičnými vlastnosťami, ktoré sú vyjadrené tým, čo hovoria a čo robia.
Na základe dôležitosti možno postavu rozdeliť do dvoch kategórií: hlavná postava a vedľajšia postava. V celom príbehu sa zvyčajne objavuje hlavná postava, stáva sa stredobodom príbehu. Udalosti, ktoré sa v príbehu dejú, sa ho vždy priamo či nepriamo týkajú. Na druhej strane, u vedľajších postáv sú úlohy menej dôležité ako u hlavnej postavy, pretože nie sú plne rozvinutými postavami a ich úlohy v príbehu sú len na podporu vývoja hlavnej postavy.

\subsection*{Prostredie}
Prostredie diela sa vzťahuje na prírodnú a umelú scenériu, v ktorom postavy žijú a pohybujú sa. Znamená to, že všetko, čo súvisí s prostredím, ako napr. denné svetlo, stromy, zvieratá, spoločnosť, popisované zvuky, pachy a počasie sú súčasťou prostredia. Dejiskom diela je opis predmetov a fyzického vzhľadu miesta, kde sa príbeh odohráva.
Popis prostredia pomáha pri vytváraní dôveryhodnosti; môže pomôcť vysvetliť postavy aj situáciu; môže prispieť k atmosfére alebo prevládajúcej nálade; môže byť aktívny v predpovedaní; môže byť symbolickým. Okrem symboliky hlavných postáv sa prostredie používa aj ako prostriedok na posilnenie témy, znamená to, že prostredie sa považuje za dôležitú úlohu v príbehu a analýze.

\todo{hlavná dejová línia vs. vedľajšie, vzťah k dĺžke zhrnutia deja,}


\section{Kľúčové dejové línie}
Pojem \textit{kľúčové dejové línie} nemožno exaktne definovať. V starovekom Grécku rozlišovali 
2 druhy deja: komédiu a tragédiu, no odvtedy sa delenie dejových línií posunulo ďalej. V minulosti sa o nájdenie rozdelenia deja pokúšalo veľa významných dramatikov, či spisovateľov, medzi nimi aj Shiller \footnote{Friedrich Schiller- nemecký dramatik (1759-1805)
} či Gozzi \footnote{Carlo Gozzi- taliansky dramatik (1720-1806)}. Na posledného menovaného nadväzuje kniha \uv{The Thirty Six Dramatic Situations} \cite{Polti:1921}, ktorá je jednou z mála kníh, ktorá sa zaoberá klasifikovaním deja. Ako z názvu vyplýva, predkladá 36 dramatických situácií, ktoré sú aj na dnešnú dobu aktuálne, no kniha udáva aj konkrétne deje pod každou dramatickou situáciou, tie sú však z väčšiny zastaralé a v dnešnej dobe nepoužiteľné.

Ľudia odpradávna hľadajú idey pre vymýšľanie nových kníh, či filmov, preto možno hľadať kľučové dejové línie najmä v knihách, ktoré sa zaoberajú inšpirovaním o témach a dejových líniách, o ktorých by mohli ich čitateľa písať. 

Jednou z takýchto kníh je \uv{Plotto} \cite{Cook:1928}, ktorá mala autorom pomôcť \textit{poskladať si} príbeh. Toto dielo prezentuje tisíce veľmi podrobných dejových línií, tzv. \textit{Masterplot}. Každá táto dejová línia pozostáva z 3 \textit{klauzúl}. Kde klauzula A reprezentuje druh protagonistu, klauzula B uvádza základnú myšlienku deja a klauzula C ukončuje dej. Keďže B klauzuly sú veľmi všeobecné, každá B klauzula pozostáva z desiatok podrobnejších dejov. Tieto podrobné deje, môžu na seba rôzne navzájom nadväzovať a vytvoriť tak celkom konkrétny podrobný dej. Celkovo počet dejov, ktorý sa týmto mechanizmom dokáže vytvoriť možno rátať v státisícoch.

Ďalšou knihou z tejto kategórie je \uv{20 Master Plots: And How to Build Them} \cite{Tobias:1993} v ktorej autor prednáša, ako opäť z názvu vyplýva, 20 oblastí ktorých sa môže týkať dej. Sú nimi: Hľadanie, Dobrodružstvo, Prenasledovanie, Záchrana, Únik, Pomsta, Hádanka, Rivalita, Smoliar, Pokušenie, Metamorfóza, Transformácia, Dozrievanie, Láska, Zakázaná láska, Obetovanie, Objavenie, Úbohý prebytok, Vzostup a Zostup. 


\section{Extrakcia dát}
Dáta sú nevyhnutnou súčasťou každého výskumu, či už to je akademický, marketingový alebo vedecký. Ľudia chcú zbierať a analyzovať údaje z viacerých webových stránok. Rôzne webové stránky, ktoré patria do konkrétnej kategórie zobrazuje informácie v rôznych formátoch. Údaje môžu byť rozložené na viacerých stránkach a v rôznych sekciách. Väčšina webových stránok neumožňuje uložiť kópiu údajov, zobrazených na ich webových stránkach do vášho miestneho úložiska. \textit{Web Scraping} je technikou extrakcie neštruktúrovaných údajov z webových stránok a taktiež transformáciou týchto údajov do štruktúrovaných údajov, ktoré možno uložiť do zrozumiteľnej štruktúry ako napr. tabuľka, databáza alebo súbor vo formáte CSV. 

Proces získavania údajov z internetu možno rozdeliť do dvoch po sebe nasledujúcich krokov; získavanie webových zdrojov a následné extrahovanie požadovaných informácií zo získaných údajov. Webové údaje sa bežne extrahujú pomocou protokolu HTTP (Hypertext Transfer Protocol) alebo cez web prehliadač. To sa buď vykonáva manuálne pomocou užívateľa alebo automaticky robotom alebo webovým prehliadačom. Konkrétne, program začína vytvorením HTTP požiadaviek na získanie zdrojov z cieľovej webovej stránky. Po úspešnom prijatí a spracovaní žiadosti cieľovou webovou stránkou sa požadovaný zdroj získa z webovej lokality a potom sa odošle späť do programu. Zdroj môže byť vo viacerých formátoch, ako sú webové stránky vytvorené z HTML, dátové kanály vo formáte XML alebo JSON alebo multimediálne dáta, akými sú obrázky, audio alebo video súbory. Po stiahnutí webových údajov proces extrakcie pokračuje v analýze, preformátovaní a usporiadaní údajov štruktúrovaným spôsobom. 

Vzhľadom na to, že sa na internete neustále generuje obrovské množstvo heterogénnych údajov, je web scraping široko uznávaný ako efektívna a výkonná technika na zber veľkých dát. 

Aj keď je web scraping účinnou technikou pri zhromažďovaní veľkých súborov údajov, je taktiež veľmi kontroverznou a môže vyvolať právne otázky súvisiace s autorskými právami a zmluvnými podmienkami. V rámci osobného použitia sa jedná o legálne využitie,  problém však nastáva v prípade komerčného využitia. Takisto, ak takýto program posiela žiadosti o získavanie údajov príliš často, je to funkčne ekvivalentné útoku odmietnutia služby (DoS - Denial-of-Service), pri ktorom môže byť vlastníkovi odmietnutý vstup a môže byť zodpovedný za vzniknuté škody podľa, pretože vlastník webovej aplikácie má majetkovú účasť na fyzickom webovom serveri, ktorý je hostiteľom aplikácie.

Táto podkapitola vychádza z publikácií \cite{Zhao:2017} a \cite{Sirisuriya:2015}.


\section{Spracovanie prirodzeného jazyka}

Spracovanie prirodzeného jazyka (NLP - angl. Natural Language Processing ) je počítačový prístup k analýze textu a je založený na súbore teórií aj na súbore technológií. Je to veľmi aktívna oblasť výskumu a vývoja v dnešnej dobe. NLP možno definovať asi nasledovne: Spracovanie prirodzeného jazyka je teoreticky motivované rozpätie výpočtových techník na analýzu a reprezentáciu prirodzene sa vyskytujúcich textov na jednej alebo viacerých úrovniach lingvistickej analýzy s cieľom dosiahnuť podobnosť s človekom pri spracovaní jazyka pre celý rad úloh alebo aplikácií.

Viaceré prvky tejto definície možno podrobnejšie rozviesť. Existuje viacero metód resp. techník, z ktorých si vybrať na vykonanie konkrétneho typu jazykovej analýzy. \uv{Prirodzene sa vyskytujúce texty} môžu byť akéhokoľvek jazyka, žánru a pod., jedinou požiadavkou je, aby boli v jazyku, ktorý používajú ľudia na komunikáciu medzi sebou. Analyzovaný text by tiež nemal byť špecifický vytvorený na účely analýzy, ale získaný zo skutočných použití. Pojem \uv{úrovne lingvistickej analýzy} odkazuje na skutočnosť, že existuje viacero typov jazykového spracovania, o ktorých je známe, že ľudia používajú pre pochopenie jazyka. NLP sa považuje za vnútornú disciplínu Umelej inteligencie (AI  - angl. Artificial Intelligence ), keďže NLP sa snaží o výsledky podobné človeku, je vhodné považovať NLP za disciplínu AI.

Cieľom NLP, ako je uvedené vyššie, je dosiahnuť spracovanie ľudského jazyka. Výber slova \uv{spracovanie} je veľmi zámerný a nemal by sa nahradiť výrazom \uv{pochopenie}. Oblasť NLP bola pôvodne označovaná ako Porozumenie prirodzeného jazyka (NLU - angl. Natural Language Understanding ) v začiatkoch AI, a keďže dodnes nebol jazyk skutočne \uv{pochopený}, tak sa NLU považuje za cieľ NLP. 

Úplný NLU systém by bol schopný:

\begin{enumerate}
    \item \textbf{Parafrázovať vstupný text}
    \item \textbf{Preložiť text do iného jazyka}
    \item \textbf{Odpovedať na otázky z textu}
    \item \textbf{Vyvodzovať závery z textu}
\end{enumerate}

Zatiaľ čo NLP urobilo vážne zásahy do dosiahnutia cieľov 1 až 3, skutočnosť, že NLP systémy nedokážu samy o sebe vyvodzovať závery z textu, cieľom NLU stále zostáva NLP \cite{Liddy:2001} .

\subsection{Klasifikácia}

Námety pre túto kapitolu vychádzali z \cite{Herrera:2016}. Klasifikácia (classification) je jednou z najpopulárnejších tém Spracovania prirodzeného jazyka a Hĺbkovej analýzy dát (Data Mining). Je to zvyčajne prediktívna úloha vedená pomocou techník učenia pod dohľadom (supervised learning). Klasifikácia má za cieľ naučiť sa z \textit{označkovaných} vzorov model schopný predpovedať značky (labels) pre budúce, nikdy predtým nevidené, ukážky údajov.

Sada atribútov v súbore údajov klasifikácie (classification dataset) je rozdelená do dvoch podmnožín. Tá prvá obsahuje vstupné vlastnosti (features), premenné, ktoré budú fungovať ako prediktory. Druhá podmnožina obsahuje výstupné atribúty, takzvané \textit{značky} (labels), ktoré sú priradené pre každú inštanciu. Klasifikačné algoritmy indukujú model analyzujúci koreláciu medzi vstupnými vlastnosťami a výstupnými značkami. Keď sa získa natrénovaný model, môže byť použitý na spracovanie množiny vstupných vlastností nových vzoriek údajov a tým získať predikciu značiek. V závislosti od povahy druhej podmnožiny atribútov, ktorá obsahuje značky, možno identifikovať niekoľko druhov klasifikačných problémov, v závislosti od počtu výstupov a ich typov. 

Medzi základné patrí: 
\begin{itemize}
    \item \textbf{Binárna klasifikácia} (Binary Classification) Toto je najjednoduchší klasifikačný problém, ktorému možno čeliť. Inštancie v binárnej dátovej sade majú iba jeden výstupný atribút a môže mať iba dve rôzne hodnoty. Tieto sú zvyčajne známe ako pozitívne a negatívne, ale možno ich interpretovať aj ako pravda a nepravda, 1 a 0 alebo akákoľvek iná kombinácia dvoch hodnôt. Klasický príklad tejto  úlohy je filtrovanie spamu, pri ktorom sa klasifikátor učí zo správ obsah, ktorý možno považovať za spam.
    
    \item \textbf{Viac-triedna klasifikácia} (Multi-class Classification) Viac-triedna dátová sada má tiež iba jeden výstupný atribút, ako binárna klasifikácia, ale môže obsahovať ktorúkoľvek z určitého súboru preddefinovaných hodnôt. Vo viac-triednej klasifikácií sa \textit{značky} nazývajú \textit{triedy} (class). Význam každého z týchto tried a samotná hodnota sú špecifické pre každú aplikáciu, súbor tried je však konečný a diskrétny. Jedným z najznámejších príkladov klasifikácie viacerých tried je identifikácia druhov dúhovky, kde sa klasifikátor učí, ako klasifikovať nové inštancie do zodpovedajúcej rodiny (triedy). Mnoho viac-triednych klasifikačných algoritmov závisí na binarizácií, metóda, ktorá iteratívne trénuje binárny klasifikátor pre každú triedu zvlášť. Viac-triednu klasifikáciu možno považovať za zovšeobecnenie binárnej klasifikácie. Výstup je len jeden, ale môže nadobudnúť akúkoľvek hodnotu, zatiaľ čo v binárnom prípade je obmedzený na podmnožinu dvoch hodnôt.
    
    \item \textbf{Viac-značková klasifikácia} (Multi-label Classification) Na rozdiel od dvoch predchádzajúcich klasifikačných modelov, vo viac-značkovej klasifikácií má každá z inštancií údajov priradený vektor výstupov, nie iba jednu hodnotu. Dĺžka tohto vektora je pevná a má dĺžku podľa počtu značiek v dátovej sade. Každý prvok vektora je binárna hodnota označujúca, či príslušná značka je alebo nie je relevantná pre vzorku. Aktívnych môže byť niekoľko značiek naraz. Viac-značková klasifikácia sa v súčasnosti používa v mnohých oblastiach, z ktorých väčšina súvisí na automatické označovanie zdrojov zo sociálnych médií, akými sú obrázky, hudba, video, správy, či blogové príspevky. Algoritmy použité na túto úlohu musia byť schopné vytvoriť niekoľko predpovedí naraz, či už ide o transformáciu pôvodných dátových sád alebo ich prispôsobenie na existujúce binárne/viac-triedne klasifikačné algoritmy.
\end{itemize}


\section{Klasifikačné metódy}
Kapitoly týkajúce sa klasifikačných metód sú prevzaté z publikácie \cite{Li:2021}. V ére informačnej explózie môže byť zdĺhavé a náročné spracovávať a klasifikovať veľké množstvo textových údajov manuálne. Okrem toho môže byť presnosť manuálnej klasifikácie textu ovplyvnená ľudskými faktormi, akými sú únava a neodbornosť. Je preto žiaduce použiť metódy strojového učenia na automatizáciu procesu klasifikácie textu, aby boli výsledky spoľahlivejšie a menej subjektívne. Okrem toho to môže tiež pomôcť zvýšiť efektivitu vyhľadávania informácií a zmierniť problém informačného preťaženia.

Textové údaje sa líšia od číselných, obrazových alebo signálových dát, vyžadujú NLP   techniky pre ich spracovanie. Prvým dôležitým krokom je predspracovanie textových údajov
pre model. Tradičné modely zvyčajne potrebujú získať dobré vlastnosti (features) umelými metódami a potom ich klasifikovať pomocou klasických algoritmov strojového učenia. Preto účinnosť týchto metóda je do značnej miery obmedzená extrakciou vlastností (feature extraction). Preto väčšina výskumných prác zameraných na klasifikáciu textu sú zamerané na hlboké neurónové siete, čo sú prístupy založené na údajoch (data-driven approaches) s vysokou výpočtovou náročnosťou.

Klasifikáciou textu sa myslí extrahovanie prvkov z nespracovaných textových údajov a predpovedanie kategórie textových údajov na základe takýchto vlastnosti (features). Množstvo modelov bolo navrhnutých za posledných niekoľko desaťročí na klasifikáciu textu. Z tradičných modelov, Naïve Bayes (NB) bol prvým použitým modelom na túto úloha. Potom sa používajú generické klasifikačné modely, ako napríklad K-Nearest Neighbor (KNN), Support Vector Machine (SVM) a Random Forest (RF). V poslednej dobe eXtreme Gradient Boosting (XGBoost) a Light Gradient Boosting Machine (LightGBM) majú potenciál poskytnúť vynikajúci výkon. Modely hlbokého učenia takisto išli veľmi do popredia, odkedy bola použitá konvolučná neurónová sieť po prvý krát na klasifikáciu textu. Aj keď nie je špeciálne navrhnutý na prácu s textom, je Bidirectional Encoder Representation from Transformers (BERT) široko používaný pri návrhu modelov pre klasifikáciu textu, vzhľadom na jeho efektivitu vo viacerých dátových sadách zameraných na textovú klasifikáciu.

\subsection*{Naïve Bayes}
Naïve Bayes (NB) je najjednoduchší a najrozšírenejší model založený na použití Bayesovho teorému. Algoritmus NB primárne využíva podmienenú pravdepodobnosť. Výhodou NB je, že na odhad parametrov potrebných na klasifikáciu vyžaduje len malý počet tréningových dát. Parametre NB sú menej citlivé na chýbajúce údaje a algoritmus je jednoduchý. Predpokladá však, že vlastnosti sú na sebe nezávislé. Keď počet vlastnosti je veľký, alebo korelácia medzi vlastnosťami je významná, výkon NB klesá. Napriek "naivnému" dizajnu a zjednodušeným predpokladom je NB široko používaný aj v zložitých problémoch.

\subsection*{K-najbližších susedov}
Jadrom algoritmu k-najbližších susedov (angl. KNN - K-Nearest Neighbors) je klasifikácia neoznačenej vzorky nájdením kategórie s najväčším počtom vzoriek na \texit{k} najbližších vzorkách. Je to jednoduchý klasifikátor bez nutnosti vytvárania modelu a môže znížiť zložitosť pomocou jednoduchého procesu získavania KNN. Avšak v dôsledku pozitívnej korelácie medzi časovou/priestorovou zložitosťou modelu a množstvom údajov, je algoritmus KNN na rozsiahlych dátových sadách nezvyčajne pomalý. KNN závisí hlavne od okolitých susedných vzoriek a pre dátové sady s väčším prekrytím tried  je vhodnejší ako iné metódy.

\subsection*{Support Vector Machine}
Prístupy založené na SVM menia klasifikáciu textu na viaceré úlohy binárnej klasifikácie. SVM sa snaží vytvárať optimálnu nadrovinu vo vektorovom priestore pre maximalizáciu vzdialenosti medzi triedami, a určiť vzdialenosť hranice kategórie v smere kolmom na nadrovinu čo najväčšiu, čo bude mať za následok nižšiu chybovosť klasifikácie. Využíva predchádzajúce znalosti na vytvorenie vhodnejšej štruktúry a rýchlejšie štúdium. SVM dokáže vyriešiť vysokorozmerné a nelineárne problémy. Má vysokú generalizačnú schopnosť, ale je citlivý na chýbajúce údaje.

\subsection*{Rozhodovacie stromy}
Rozhodovacie stromy (DT- Decision Trees) sú metódy učenia s učiteľom stromovej štruktúry a sú konštruovaný rekurzívne, odrážajú myšlienku \texit{rozdeľuj a panuj}. Rozhodovacie stromy môžu byť všeobecne rozdelené do dvoch odlišných etáp: stavba stromov a prerezávanie stromov. Začína sa na koreňovom uzly kde testuje vzorky údajov, a rozdeľuje dátovú sadu do rozdielnych podmnožín podľa rôznych výsledkov. Podmnožiny dátovej sady tvoria synovské uzly a každý listový uzol v rozhodovacom strome predstavuje kategóriu. Konštrukcia rozhodovacieho stromu má určiť koreláciu medzi triedami a atribútmi, ktoré sa ďalej využívajú na predpovedanie kategórie neznámych budúcich typov. DT je ľahké pochopiť a interpretovať. Z pozorovaneho modelu je ľahké odvodiť zodpovedajúci logický výraz z vygenerovaného rozhodovacieho stromu.

\subsection*{Neurónové siete}
Hlboké neurónové siete pozostávajú z umelých neurónových sietí, ktoré simulujú ľudský mozog, aby sa automaticky učil vysokoúrovňové vlastnosti z údajov, ktoré dosahujú lepšie výsledky ako tradičné modely spomenuté vyššie. Konvolučné neurnové siete (CNN - Convolutional Neural Networks ) môže súčasne aplikovať konvolúcie definované rôznymi jadrami na viaceré časti sekvencie. Preto sa CNN používajú na mnohé úlohy NLP vrátane klasifikácie textu. Pre klasifikáciu textu sa vyžaduje, aby bol text reprezentovaný ako vektor. Najskôr sú všetky vektory slov vstupného textu spojené do matice. Matica sa potom privádza do konvolučnej vrstvy, ktorá obsahuje niekoľko filtrov s rôznymi rozmermi. Nakoniec výsledok konvolučnej vrstvy prechádza cez rozhodovaciu vrstva a zreťazí všetky výsledky, aby sa získala konečná vektorová reprezentácia, ktorá určuje výslednú triedu. 
Objavenie BERT, ktorý dokáže generovať kontextové slovné vektory, je významným bodom rozvoja klasifikácie textu a iných technológií NLP. Mnoho výskumníkov študovalo modely klasifikácie textu založené na BERT, ktorý dosahuje lepšie výsledky ako vyššie uvedené modely vo viacerých úlohách NLP vrátane klasifikácie textu.

CNN extrahuje vlastnosti z textových vektorov prostredníctvom konvolučných jadier. Počet vlastnsoti zachytených konvolučným jadrom súvisí s jeho veľkosťou. CNN je hlboká dosť na to, že teoreticky dokáže zachytiť vlastnosti na veľké vzdialenosti. Z dôvodu nedostatočnej optimalizácie
metódy pre parametre hlbokej siete a stratu informácií o polohe v dôsledku združovania
vrstva, hlbšia vrstva neprináša výrazné zlepšenie.

Transformer zaobchádza so vstupným textom ako s plne prepojeným grafom, je schopný paralelných výpočtov a je vysoko efektívny pri extrakcii vlastnosti medzi rôznymi slovami pomocou \texit{seba-pozornosti}. Mechanizmus pozornosti v Transformeri je však náročný na výpočty, najmä pri práci s dlhými sekvenciami. Celkovo je Transformer lepšou voľbou na klasifikáciu textu. 
Hlboké učenie pozostáva z viacerých skrytých vrstiev v neurónovej sieti s vyššou úrovňou zložitosti a môže byť trénovaný na neštruktúrovaných dátach. Hlboké učenie sa dokáže naučiť jazykové vlastnosti a dokáže zvládať aj abstraktnejšie jazykové prvky založené na slovách a vektoroch na vysokej úrovni. Architektúra hlbokého učenia sa vie naučiť reprezentovať vlastnosti priamo zo vstupu bez príliš veľkého manuálneho zásahu a predchádzajúcich znalostí. Technológia hlbokého učenia je však metódou založenou na dátach, ktorá si vyžaduje obrovské množstvo údajov na dosiahnutie vysokého výkonu. 

Predtrénované jazykové modely sa efektívne učia globálnu reprezentáciu sémantiky a výrazne posilňujú NLP úlohy, vrátane klasifikácie textov. Vo všeobecnosti používa metódy bez dozoru na automatické dolovanie sémantických znalostí a následné vytvorenie predtrénované ciele tak, aby sa stroje naučili chápať sémantiku

\section{Transformer}

\subsection{Bert}

\subsection{Fine-tuning}


\section{Spracovanie dlhých reťazcov}


\section{Metriky vyhodnocovania}
Táto podkapitola je prebraná z \cite{Herrera:2016}. Výstup akéhokoľvek klasifikátora s viacerými značkami pozostáva z predpovedaného vektora značiek pre každú testovaciu inštanciu. Pri práci v tradičnom scenári s iba jednou triedou ako výstupom,  predpoveď môže byť správna alebo nesprávna, naproti tomu predpovede viacerých značiek môžu byť úplne správne, čiastočne správne/nesprávne (v rôznych stupňoch) alebo úplne nesprávne.
Použitie rovnakých metrík používaných v tradičnej klasifikácii je možné, ale zvyčajne prehnane prísne. To je dôvod na použitie špecifických hodnotiacich metrík, kde sa berú do úvahy aj prípady medzi týmito dvoma extrémami.
V súčasnosti je v literatúre definovaných viac ako dvadsať rôznych výkonnostných metrík pre viac-značkovú klasifikáciu. Všetky metriky vyhodnocovania viacerých značiek možno zoskupiť podľa dvoch kritérií:
\begin{itemize}
\item \textbf{Podľa toho ako sa  predpoveď vypočítava}: Meranie môže byť vykonané inštanciou alebo pomocou
značiek, čím sa získajú dve rôzne skupiny metrík:
    \begin{itemize}
    \item \textbf{Metriky založené na príkladoch}: Tieto metriky sa počítajú samostatne pre každú inštanciu a potom sú  spriemerované počtom vzoriek.
    \item \textbf{Metriky založené na značkách}: Na rozdiel od predchádzajúcej skupiny metriky založené na značkách sa vypočítajú nezávisle pre každú značku pred ich spriemerovaním. Pritom možno použiť dve rôzne stratégie:
        \begin{itemize}
        \item \textbf{Makro-priemerovanie}: Metrika sa vypočítava individuálne pre každú značku a
        výsledok je spriemerovaný počtom značiek.
        \item \textbf{Mikro-priemerovanie}: Počítadlá trafených a netrafených predpovedí pre každú značku sú najskôr agregované a potom sa metrika vypočíta iba raz.
        \end{itemize}
    \end{itemize}
\item \textbf{Podľa toho ako sa výsledok sprostredkuje}: Výstup produkovaný viac-značkovým klasifikátorom môže byť binárna bipartícia značiek alebo poradie značiek. Niektoré z nich poskytujú oboje výsledky.
    \begin{itemize}
    \item \textbf{Binárna bipartícia}: Binárne bipartícia je vektor označujúci 0 a 1, ktoré indikujú, že značka je relevantná pre spracovanú vzorku. Existujú metriky, ktoré fungujú nad týmito bipartíciami a počítajú počet skutočne pozitívných (angl. True Positive - TP), skutočne negatívných (angl. True Negative - TN), falošne pozitívných (angl. False Positive - FP) a falošne negatívných (angl. False Negative - FN) vzoriek.
    \item \textbf{Poradie značiek}: Výstupom je zoznam značiek zoradených podľa nejakej relevantnosti. Binárnu bipartíciu možno z tohoto poradia získať použitím prahu, zvyčajne daný samotným klasifikátorom. Existujú však metriky, ktoré namiesto toho pracujú s nespracovaným poradím na výpočet vyhodnotenia.
    \end{itemize}
\end{itemize}

\subsection*{Hammingova strata} 
Hammingova strata (angl. Hamming loss) je jednou z metrík založených na príkladoch, je pravdepodobne najbežnejšie používanou metrikou viac-značkovej klasifikácie, asi aj pretože je ľahké ju vypočítať ako možno vidieť v rovnici \eqref{rovnica1}. Operátor \(\bigtriangleup\) vráti symetrický rozdiel medzi \(Y_{i}\), reálnym vektorom značiek i-tej inštancie a tým predpovedaným \(Z_{i}\). Operátor \(|r|\) počíta počet 1 v tomto rozdiele, inými slovami počet nesprávnych predpovedí. Celkový počet chýb v \(n\) inštanciách sa agreguje a potom normalizuje počtom značiek \(k\) a počtom inštancií.

\subsection*{Presnosť} 
Narozdiel od anglického jazyka, kde na slovo \textit{presnosť} majú 2 slová \textit{Accuracy} a \textit{Precision}, v našich jazykoch nanešťastie nevieme tieto dve slová odlíšiť, no predsa je medzi nimi rozdiel, preto v nasledujúcich odstavcoch budú použité dané anglické výrazy.  

Vo viac-značkovej oblasti je \textit{Accuracy} definovaná ako \eqref{rovnica2} pomer medzi hodnotami počtu správne predpovedaných značiek a celkového počtu značiek v oboch vektoroch. Metrika je počítaná pre každú inštanciu a potom je spriemerovaná ako všetky metriky založené na príkladoch.

\textit{Precision} \eqref{rovnica3}, naproti tomu, sa považuje za jednu z intuitívnejších metrík na hodnotenie viac-značkových klasifikátorov. Vypočíta sa ako podiel medzi počtom skutočne pozitívnych značiek a celkovým počtom pozitívnych značiek. Teda môže byť interpretovaná ako percento predpovedaných značiek, ktoré sú skutočne relevantné pre danú inštanciu. Táto metrika sa zvyčajne používa v spojení s funkciou \textit{Recall} (v tomto prípade je preklad ťažko interpretovať) \eqref{rovnica4}, ktorá vracia percento správne predpovedaných značiek spomedzi všetkých skutočne relevantných značiek, teda pomer správnych značiek je výstupom klasifikátora.

\subsection*{F-skóre}
Spoločné používanie \textit{Precision} a \textit{Recall} je tak bežné, že je definovaná metrika, ktorá ich kombinuje. Je známa ako F-skóre (angl. F-measure/ F-score) \eqref{rovnica5} a vypočíta sa ako harmonický priemer týchto dvoch. Týmto spôsobom je vyvážená miera toho, koľko relevantných značiek je predpovedaných a koľko predpovedaných značiek je relevantných.

\subsection*{Priemerná presnosť}
Všetky metriky založené na príkladoch popísané vyššie fungujú cez binárnu bipartíciu značiek, takže potrebujú sadu značiek ako výstup z klasifikátora. Naproti tomu, nasledujúca metrika potrebujú poradie značiek, takže buď stupeň spoľahlivosti alebo pravdepodobnosť príslušnosti každej zo značiek je potrebná na jej výpočet.

Metrika priemerná presnosť (angl. Average precision) \eqref{rovnica6} určuje pre každú značku v inštancií, podiel relevantných značiek, ktoré sú v predpokladanom poradí nad ňou. Cieľom tejto metriky je zistiť, koľko pozícií treba v priemere skontrolovať, predtým, než sa nájde nerelevantné označenie. Čím väčšia je hodnota tejto metriky, tým lepší je výkon klasifikátora.
V rovnici \eqref{rovnica6} je \(rank(x_{i},l)\) definovaná ako funkcia ktorá pre inštanciu \(x_{i}\) a príslušnú značku \(l \), ktorej poloha je známa, vráti stupeň spoľahlivosti.


\subsection*{Metriky založené na značkách}
Všetky výkonnostné metriky vymenované v predchádzajúcich častiach sa vyhodnocujú jednotlivo pre každú inštanciu a potom sú spriemerované počtom inštancií. Preto má každá vzorka údajov v konečnom výsledku rovnakú váhu. Na druhej strane, metriky založené na značkách možno vypočítať pomocou dvoch rôznych stratégií spriemerovania. Tieto sú zvyčajne známe ako makro-priemerovanie a mikro-priemerovanie. Ktorákoľvek z metrík získaných z binárnej bipartície, ako \textit{Precision}, \textit{Recall} či F-skóre je možné vypočítať pomocou týchto stratégií.

V prístupe makro-priemerovania \eqref{rovnica7} sa metrika vyhodnocuje raz za značku pomocou akumulovaného počítadlá a priemer sa potom získa vydelením počtom značiek. Takto je každej značke priradená rovnaká váha, či už je častá alebo zriedkavá. Naopak, stratégia mikro-priemerovania \eqref{rovnica8} najskôr pripočítava počítadlá pre všetky značky a potom vypočíta metriku iba raz. Preto prínos každej značky vo finále nie je rovnaký.


\begin{equation}
	HammingLoss =  \dfrac{1}{n} \dfrac{1}{k} \sum_{i=1}^{n} | Y_{i} \bigtriangleup Z_{i} | \label{rovnica1}
\end{equation}
\begin{equation}
	Accuracy =  \dfrac{1}{n}  \sum_{i=1}^{n} \dfrac{| Y_{i} \cap Z_{i} |} {| Y_{i} \cup Z_{i} |} = \dfrac{TP + TN}{TP + TN + FP + FN} \label{rovnica2}
\end{equation}
\begin{equation}
	Precision =  \dfrac{1}{n}  \sum_{i=1}^{n} \dfrac{| Y_{i} \cap Z_{i} |} {| Z_{i} |} = \dfrac{TP}{TP + FP} \label{rovnica3}
\end{equation}
\begin{equation}
	Recall =  \dfrac{1}{n}  \sum_{i=1}^{n} \dfrac{| Y_{i} \cap Z_{i} |} {| Y_{i} |} = \dfrac{TP}{TP + FN} \label{rovnica4}
\end{equation}
\begin{equation}
	\textit{F-score} =  2* \dfrac{Precision*Recall}{Precision+Recall} =  \dfrac{TP}{TP + \dfrac{1}{2} (FP + FN)} \label{rovnica5}
\end{equation}
\begin{equation}
	AveragePrecision = \dfrac{1}{n}  \sum_{i=1}^{n} \dfrac{1}{| Y_{i} |} \sum_{y \in Y_{i}} 
	\dfrac{| \{ y^{'}|rank(x_{i},y^{'}) \leq rank(x_{i},y), y^{'} \in  Y_{i} \}|} {rank(x_{i},y)} \label{rovnica6}
\end{equation}
\begin{equation}
	MacroMetric =  \dfrac{1}{k} \sum_{l \in L} EvalMetric(TP_{l},FP_{l},TN_{l},FN_{l})  \label{rovnica7}
\end{equation}
\begin{equation}
	MicroMetric =  EvalMetric( \sum_{l \in L} TP_{l}, \sum_{l \in L} FP_{l}, \sum_{l \in L} TN_{l}, \sum_{l \in L} FN_{l} )  \label{rovnica8}
\end{equation}


\noindent\(n\) = počet inštancií \newline
\(k\) = počet značiek  \newline
\(Y_{i}\) = reálny vektor značiek i-tej inštancie  \newline
\(Z_{i}\) = predpovedaný vektor značiek i-tej inštancie \newline
\(\bigtriangleup\) = symetrický rozdiel  \newline
\(|r|\) = počet nesprávnych predpovedí \newline
\(TP\) = True Positive \newline
\(TN\) = True Negative \newline
\(FP\) = False Positive \newline
\(FN\) = False Negative \newline
\(L\) = všetky značky v dátovej sade  \newline

\section{Doterajšie práce}


\chapter{Návrh systému} 
\label{navrh}

\section{Motivácia}


\section{Schéma systému}


\section{Dáta}

\subsection{Sťahovanie dát}

\subsection{Získavanie dát}

\subsection{Popis dát}


\section{Analýza dát}


\section{Zhrnutie návrhu}


\chapter{Implementácia systému}
\label{implementacia}

\section{Architektúra projektu}


\section{Sťahovanie/ výber dát}


\section{Klasifikácia}


\section{Možné vylepšenia}


\chapter{Experimentálne vyhodnotenie a diskusia}
\label{experimenty}


\section{Dataset}


\section{Maximálna presnosť}


\section{Využitie}


\section{Príklad}


\chapter{Záver}
\label{zaver}





%===============================================================================
