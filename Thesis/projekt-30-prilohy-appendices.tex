\chapter{Obsah priloženého pamäťového média}

Na pamäťovom médiu sa nachádza:
\begin{itemize}
    \item Zložka {\tt thesis} so zdrojovými súbormi a ostatnými dátami pre vytvorenie technickej správy.
    \item Zložku {\tt data}, ktorá obsahuje potrebné skripty pre pripravenie dátových sád:
    \begin{itemize}
        \item Zložku {\tt goodreads} obsahujúca skripty pre extrakciu dát z Goodreads: {\tt download.py} pre sťahovanie HTML súborov,  {\tt extract\_book\_info.py} a {\tt extract\_fuzzy\_book\_info.py} pre extrahovanie kľúčových údajov a vytvorenie dátovej sady.
        \item Zložku {\tt processed\_dataset} -- spracovaný dataset z IMDb, aby sa nimi dal model trénovať.
        \item {\tt extract\_keywords\_imdb.py} -- súbor extrahujúci kľúčové dáta z HTML súborov potrebné pre vytvorenie dátovej sady.
         \item {\tt prepare\_imdb\_data.py} -- súbor, ktorý pripravujú dátovú sadu na tréning.
         \item {\tt plot\_keywords.txt} -- súbor, obsahujúci zoznam všetkých kľúčových dejových línií.
    \end{itemize}
    \item Zložku {\tt classifier}, ktorá obsahuje potrebné skripty pre trénovanie a prácu s klasifikátorom:
    \begin{itemize}
        \item Zložku {\tt best\_trained\_roberta}, ktorá obsahuje najlepšie natrénovaný model klasifikátora.
        \item {\tt trainer\_imdb.py} -- súbor, ktorý zabezpečuje tréning na dátovej sade z IMDb.
        \item {\tt load\_model\_imdb.py} -- súbor, ktorý načíta model zo zložky {\tt best\_trained\_roberta} a vyhodnotí ho na validačnej testovacej sade.
        \item {\tt compare\_plots.py} -- súbor, ktorý porovnáva dva zhrnutia deja pomocou metrík podobnosti.
        \item {\tt predict\_plot.py} -- súbor, ktorý určí kľúčové dejové línie a aj predpovede pre zadané zhrnutie deja.
    \end{itemize}
    \pagebreak
    \item {\tt thesis.pdf} -- súbor, obsahujú technickú správu vo formáte PDF určenú na čítanie.
    \item {\tt thesis\_print.pdf} -- súbor, obsahujú technickú správu vo formáte PDF určenú na tlač.
    \item {\tt plagat.pdf} -- súbor, obsahujúci plagát popisujúci túto prácu.
    \item {\tt requirements.txt} -- súbor, obsahujúci informáciu o potrebných knižniciach pre spustenie skriptov.
    \item {\tt README} -- súbor, obsahujúci obsah pamäťového média a manuál.
\end{itemize}

\chapter{Manuál}

Pre spustenie všetkých skriptov je nutné sa uistiť, že sú k dispozícií knižnice popísané v {\tt requirements.txt}. 


\chapter{Zoznam kľúčových dejových línií} \label{priloha-kdl}

\begin{minipage}[t]{.5\textwidth}
    \begin{itemize}
        \item Doppelganger 
        \item Memory Loss
        \item Manipulation
        \item Near Death Experience
        \item Mind Control
        \item Rise To Power
        \item Rise And Fall
        \item Saving The World
        \item Loss Of Father
        \item Quest
        \item Pursuit
        \item Rescue
        \item Revenge
        \item Riddle
        \item Rivalry
        \item Underdog
        \item Temptation
        
    \end{itemize}   
\end{minipage}%
\begin{minipage}[t]{.5\textwidth}
        \begin{itemize}
        \item Guilt
        \item Secret
        \item Justice
        \item Prejudice
        \item Racism
        \item Abuse Of Power
        \item Power Struggle
        \item Time Travel
        \item Supernatural Power
        \item Time Loop
        \item Alternate Reality
        \item Good Versus Evil
        \item One Against Many
        \item One Man Army
        \item Alternate History
        \item Romantic Rivalry
        \item Race Against Time
    \end{itemize}
\end{minipage}%
\pagebreak

\begin{minipage}[t]{.5\textwidth}
       \begin{itemize}
        \item Detention
        \item Shot To Death
        \item Gunfight
        \item Flashback
        \item Escape
        \item Beating
        \item Betrayal
        \item Drunkenness
        \item Fear
        \item Drinking
        \item Deception
        \item Competition
        \item Dream
        \item Lie 
        \item Vandalism
        \item Conspiracy
        \item Corruption
        \item Courage
        \item Ambush
        \item Paranoia
        \item Faith
        \item Trauma
        \item Self Sacrifice
        \item Childhood Memory
        \item Moral Dilemma
        \item Erased Memory
        \item Brainwashing
    \end{itemize}
\end{minipage}
\begin{minipage}[t]{.5\textwidth}
       \begin{itemize}
        \item Friendship
        \item Family Relationships
        \item Love
        \item Marriage
        \item Pregnancy
        \item Infidelity
        \item Jealousy
        \item Divorce
        \item Love Triangle
        \item Marriage Proposal
        \item Unrequited Love  
        \item Love At First Sight
        \item Death
        \item Murder
        \item Violence
        \item Fight
        \item Kidnapping
        \item Chase
        \item Suicide
        \item Battle
        \item Investigation
        \item Blackmail
        \item Metamorphosis
        \item Transformation
        \item Forbidden Love
        \item Sacrifice
        \item Discovery
    \end{itemize}
\end{minipage}


\chapter{Zhrnutia dejov} \label{priloha-zhntuia}

Zhrnutie deja knihy od Erich Maria Remarque - Na západe nič nové:\\*

\noindent''\textit{The book tells the story of Paul Bäumer, a German soldier who—urged on by his school teacher—joins the German army shortly after the start of World War I. Bäumer arrives at the Western Front with his friends and schoolmates (Tjaden, Müller, Kropp and a number of other characters). There they meet Stanislaus Katczinsky, an older soldier, nicknamed Kat, who becomes Paul's mentor. While fighting at the front, Bäumer and his comrades have to engage in frequent battles and endure the dangerous and often dirty conditions of warfare. At the very beginning of the book Erich Maria Remarque says 'This book is to be neither an accusation nor a confession, and least of all an adventure, for death is not an adventure to those who stand face to face with it. It will try simply to tell of a generation of men who, even though they may have escaped shells, were destroyed by the war.' The book does not focus on heroic stories of bravery, but rather gives a view of the conditions in which the soldiers find themselves. The monotony between battles, the constant threat of artillery fire and bombardments, the struggle to find food, the lack of training of young recruits (meaning lower chances of survival), and the overarching role of random chance in the lives and deaths of the soldiers are described in detail. The battles fought here have no names and seem to have little overall significance, except for the impending possibility of injury or death for Bäumer and his comrades. Only pitifully small pieces of land are gained, about the size of a football field, which are often lost again later. Remarque often refers to the living soldiers as old and dead, emotionally drained and shaken. 'We are not youth any longer. We don't want to take the world by storm. We are fleeing from ourselves, from our life. We were eighteen and had begun to love life and the world; and we had to shoot it to pieces.' Paul's visit on leave to his home highlights the cost of the war on his psyche. The town has not changed since he went off to war; however, he finds that he does 'not belong here anymore, it is a foreign world.' He feels disconnected from most of the townspeople. His father asks him 'stupid and distressing' questions about his war experiences, not understanding 'that a man cannot talk of such things.' An old schoolmaster lectures him about strategy and advancing to Paris, while insisting that Paul and his friends know only their 'own little sector' of the war but nothing of the big picture. Indeed, the only person he remains connected to is his dying mother, with whom he shares a tender, yet restrained relationship. The night before he is to return from leave, he stays up with her, exchanging small expressions of love and concern for each other. He thinks to himself, 'Ah! Mother, Mother! How can it be that I must part from you? Here I sit and there you are lying; we have so much to say, and we shall never say it.' In the end, he concludes that he 'ought never to have come home on leave.' Paul feels glad to be reunited with his comrades. Soon after, he volunteers to go on a patrol and kills a man for the first time in hand-to-hand combat. He watches the man die, in pain for hours. He feels remorse and asks forgiveness from the man's corpse. He is devastated and later confesses to Kat and Albert, who try to comfort him and reassure him that it is only part of the war. They are then sent on what Paul calls a 'good job'. They must guard a village that is being shelled too heavily. The men enjoy themselves but while evacuating the villagers, Paul and Albert are wounded. They recuperate in a Catholic hospital and Paul returns to active duty. By now, the war is nearing its end and the German Army is retreating. In despair, Paul watches as his friends fall one by one. It is the death of Kat that eventually makes Paul careless about living. In the final chapter, he comments that peace is coming soon, but he does not see the future as bright and shining with hope. Paul feels that he has no aims left in life and that their generation will be different and misunderstood. When he finally dies at the end of the novel, the situation report from the frontline states, 'All is Quiet on the Western Front,' symbolizing the cheapness of human life in war.
}''\\*

\noindent Zhrnutie deja hry od William Shakespeare - Rómeo and Júlia:\\*

\noindent''\textit{The play, set in Verona, begins with a street brawl between Montague and Capulet supporters who are sworn enemies. The Prince of Verona intervenes and declares that further breach of the peace will be punishable by death. Later, Count Paris talks to Capulet about marrying his daughter, but Capulet asks Paris to wait another two years (then he later orders Juliet to marry Paris) and invites him to attend a planned Capulet ball. Lady Capulet and Juliet's nurse try to persuade Juliet to accept Paris's courtship. Meanwhile, Benvolio talks with his cousin Romeo, Montague's son, about Romeo's recent depression. Benvolio discovers that it stems from unrequited infatuation for a girl named Rosaline, one of Capulet's nieces. Persuaded by Benvolio and Mercutio, Romeo attends the ball at the Capulet house in hopes of meeting Rosaline. However, Romeo instead meets and falls in love with Juliet. After the ball, in what is now called the 'balcony scene', Romeo sneaks into the Capulet orchard and overhears Juliet at her window vowing her love to him in spite of her family's hatred of the Montagues. Romeo makes himself known to her and they agree to be married. With the help of Friar Laurence, who hopes to reconcile the two families through their children's union, they are secretly married the next day. Juliet's cousin Tybalt, incensed that Romeo had sneaked into the Capulet ball, challenges him to a duel. Romeo, now considering Tybalt his kinsman, refuses to fight. Mercutio is offended by Tybalt's insolence, as well as Romeo's 'vile submission,' and accepts the duel on Romeo's behalf. Mercutio is fatally wounded when Romeo attempts to break up the fight. Grief-stricken and wracked with guilt, Romeo confronts and slays Tybalt. Montague argues that Romeo has justly executed Tybalt for the murder of Mercutio. The Prince, now having lost a kinsman in the warring families' feud, exiles Romeo from Verona, with threat of execution upon return. Romeo secretly spends the night in Juliet's chamber, where they consummate their marriage. Capulet, misinterpreting Juliet's grief, agrees to marry her to Count Paris and threatens to disown her when she refuses to become Paris's 'joyful bride.' When she then pleads for the marriage to be delayed, her mother rejects her. Juliet visits Friar Laurence for help, and he offers her a drug that will put her into a deathlike coma for 'two and forty hours.' The Friar promises to send a messenger to inform Romeo of the plan, so that he can rejoin her when she awakens. On the night before the wedding, she takes the drug and, when discovered apparently dead, she is laid in the family crypt. The messenger, however, does not reach Romeo and, instead, Romeo learns of Juliet's apparent death from his servant Balthasar. Heartbroken, Romeo buys poison from an apothecary and goes to the Capulet crypt. He encounters Paris who has come to mourn Juliet privately. Believing Romeo to be a vandal, Paris confronts him and, in the ensuing battle, Romeo kills Paris. Still believing Juliet to be dead, he drinks the poison. Juliet then awakens and, finding Romeo dead, stabs herself with his dagger. The feuding families and the Prince meet at the tomb to find all three dead. Friar Laurence recounts the story of the two 'star-cross'd lovers'. The families are reconciled by their children's deaths and agree to end their violent feud. The play ends with the Prince's elegy for the lovers: 'For never was a story of more woe Than this of Juliet and her Romeo.'
}''\\*

\noindent Zhrnutie deja knihy od Anthony Burgess - Mechanický pomaranč:\\*

\noindent''\textit{Alex, a teenager living in near-future England, leads his gang on nightly orgies of opportunistic, random 'ultra-violence'. Alex's friends ('droogs' in the novel's Anglo-Russian slang, Nadsat) are: Dim, a slow-witted bruiser who is the gang's muscle; Georgie, an ambitious second-in-command; and Pete, who mostly plays along as the droogs indulge their taste for ultra-violence. Characterized as a sociopath and a hardened juvenile delinquent, Alex is also intelligent and quick-witted, with sophisticated taste in music, being particularly fond of Beethoven, or 'Lovely Ludwig Van'. The novel begins with the droogs sitting in their favorite hangout (the Korova Milkbar), drinking milk-drug cocktails, called 'milk-plus', to hype themselves for the night's mayhem. They assault a scholar walking home from the public library, rob a store leaving the owner and his wife bloodied and unconscious, stomp a panhandling derelict, then scuffle with a rival gang. Joyriding through the countryside in a stolen car, they break into an isolated cottage and maul the young couple living there, beating the husband and raping his wife. In a metafictional touch, the husband is a writer working on a manuscript called 'A Clockwork Orange', and Alex contemptuously reads out a paragraph that states the novel's main theme before shredding the manuscript. Back at the milk bar, Alex punishes Dim for some crude behaviour, and strains within the gang become apparent. At home in his dreary flat, Alex plays classical music at top volume while fantasizing of even more orgiastic violence. Alex skips school the next day. Following an unexpected visit from P.R. Deltoid, his 'post-corrective advisor', Alex meets a pair of ten-year-old girls and takes them back to his parents' flat, where he administers hard drugs and then rapes them. That evening, Alex finds his droogs in a mutinous mood. Georgie challenges Alex for leadership of the gang, demanding that they pull a 'man-sized' job. Alex quells the rebellion by slashing Dim's hand and fighting with Georgie, then in a show of generosity takes them to a bar, where Alex insists on following through on Georgie's idea to burgle the home of a wealthy old woman. The break-in starts as farce and ends in tragic pathos, as Alex's attack kills the elderly woman. His escape is blocked by Dim, who attacks Alex, leaving him incapacitated on the front step as the police arrive. Sentenced to prison for murder, Alex gets a job at the Wing chapel playing religious music on the stereo before and after services as well as during the singing of hymns. The prison chaplain mistakes Alex's Bible studies for stirrings of faith (Alex is actually reading Scripture for the violent passages). After Alex's fellow cellmates blame him for beating a troublesome cellmate to death, he agrees to undergo an experimental behaviour-modification treatment called the Ludovico Technique. The technique is a form of aversion therapy in which Alex receives an injection that makes him feel sick while watching graphically violent films, eventually conditioning him to suffer crippling bouts of nausea at the mere thought of violence. As an unintended consequence, the soundtrack to one of the films—Beethoven's Fifth Symphony—renders Alex unable to listen to his beloved classical music. The effectiveness of the technique is demonstrated to a group of VIPs, who watch as Alex collapses before a walloping bully, and abases himself before a scantily-clad young woman whose presence has aroused his predatory sexual inclinations. Though the prison chaplain accuses the state of stripping Alex of free will, the government officials on the scene are pleased with the results and Alex is released into society. Since his parents are now renting his room to a lodger, Alex wanders the streets and enters a public library where he hopes to learn a painless way to commit suicide. There, he accidentally encounters the old scholar he assaulted earlier in the book, who, keen on revenge, beats Alex with the help of his friends. The policemen who come to Alex's rescue turn out to be none other than Dim and former gang rival Billyboy. The two policemen take Alex outside of town and beat him up. Dazed and bloodied, Alex collapses at the door of an isolated cottage, realizing too late that it is the house he and his droogs invaded in the first half of the story. Because the gang wore masks during the assault, the writer does not recognize Alex. The writer, whose name is revealed as F. Alexander, shelters Alex and questions him about the conditioning. During this sequence, it is revealed that Mrs. Alexander died from the injuries inflicted during the gang-rape, and her husband has decided to continue living 'where her fragrant memory persists' despite the horrid memories. Alexander, a critic of the government, hopes to use Alex as a symbol of state brutality and thereby prevent the incumbent government from being re-elected. Eventually, he begins to realize Alex's role in the happenings of the night two years ago. One of Alexander's radical associates manages to extract a confession from Alex after removing him from F. Alexander's home and then locks him in a flatblock near his former home. Alex is then subjected to a relentless barrage of classical music, prompting him to attempt suicide by leaping from a high window. Alex wakes up in hospital, where he is courted by government officials anxious to counter the bad publicity created by his suicide attempt. With Alexander safely packed off to a mental institution, Alex is offered a well-paying job if he agrees to side with the government. As photographers snap pictures, Alex daydreams of orgiastic violence and realizes the Ludovico conditioning has been reversed: 'I was cured all right'. In the final chapter, Alex has a new trio of droogs, but he finds he is beginning to outgrow his taste for violence. A chance encounter with Pete, now married and settled down, inspires Alex to seek a wife and family of his own. He contemplates the likelihood of his future son being a delinquent as he was, a prospect Alex views fatalistically.
}''\\*

\noindent Zhrnutie deja knihy od Fyodor Dostoyevsky - Zločin a trest:\\*

\noindent''\textit{Raskolnikov, a conflicted former student, lives in a tiny, rented room in Saint Petersburg. He refuses all help, even from his friend Razumikhin, and devises a plan to murder and to rob an unpleasant elderly pawn-broker and money-lender, Alyona Ivanovna. His motivation comes from the overwhelming sense that he is predetermined to kill the old woman by some power outside of himself. While still considering the plan, Raskolnikov makes the acquaintance of Semyon Zakharovich Marmeladov, a drunkard who recently squandered his family's little wealth. He also receives a letter from his sister and mother, speaking of their coming visit to Saint Petersburg, and his sister's sudden marriage plans which they plan on discussing upon their arrival. After much deliberation, Raskolnikov sneaks into Alyona Ivanovna's apartment where he murders her with an axe. He also kills her half-sister, Lizaveta, who happens to stumble upon the scene of the crime. Shaken by his actions, Raskolnikov manages to only steal a handful of items and a small purse, leaving much of the pawn-broker's wealth untouched. Raskolnikov then flees and, due to a series of coincidences, manages to leave unseen and undetected. After the bungled murder, Raskolnikov falls into a feverish state and begins to worry obsessively over the murder. He hides the stolen items and purse under a rock, and tries desperately to clean his clothing of any blood or evidence. He falls into a fever later that day, though not before calling briefly on his old friend Razumikhin. As the fever comes and goes in the following days, Raskolnikov behaves as though he wishes to betray himself. He shows strange reactions to whoever mentions the murder of the pawn-broker, which is now known about and talked of in the city. In his delirium, Raskolnikov wanders Saint Petersburg, drawing more and more attention to himself and his relation to the crime. In one of his walks through the city, he sees Marmeladov, who has been struck mortally by a carriage in the streets. Rushing to help him, Raskolnikov gives the remainder of his money to the man's family, which includes his teenage daughter, Sonya, who has been forced to become a prostitute to support her family. In the meantime, Raskolnikov's mother, Pulkheria Alexandrovna, and his sister, Avdotya Romanovna (or Dounia) have arrived in the city. Avdotya had been working as a governess for the Svidrigaïlov family until this point, but was forced out of the position by the head of the family, Arkady Ivanovich Svidrigaïlov. Svidrigaïlov, a married man, was attracted to Avdotya's physical beauty and her feminine qualities, and offered her riches and elopement. Avdotya, having none of this, fled the family and lost her source of income, only to meet Pyotr Petrovich Luzhin, a man of modest income and rank. Luzhin proposes to marry Avdotya, thereby securing her and her mother's financial safety, provided she accept him quickly and without question. It is for these very reasons that the two of them come to Saint Petersburg, both to meet Luzhin there and to attain Raskolnikov's approval. Luzhin, however, calls on Raskolnikov while he is in a delirious state and presents himself as a foolish, self-righteous and presuming man. Raskolnikov dismisses him immediately as a potential husband for his sister, and realizes that she only accepted him to help her family. As the novel progresses, Raskolnikov is introduced to the detective Porfiry, who begins to suspect him for the murder purely on psychological grounds. At the same time, a chaste relationship develops between Raskolnikov and Sonya. Sonya, though a prostitute, is full of Christian virtue and is only driven into the profession by her family's poverty. Meanwhile, Razumikhin and Raskolnikov manage to keep Avdotya from continuing her relationship with Luzhin, whose true character is exposed to be conniving and base. At this point, Svidrigaïlov appears on the scene, having come from the province to Petersburg, almost solely to seek out Avdotya. He reveals that his wife is dead, and that he is willing to pay Avdotya a vast sum of money in exchange for nothing. She, upon hearing the news, refuses flat out, suspecting him of treachery. As Raskolnikov and Porfiry continue to meet, Raskolnikov's motives for the crime become exposed. Porfiry becomes increasingly certain of the man's guilt, but has no concrete evidence or witnesses with which to back up this suspicion. Furthermore, another man admits to committing the crime under questioning and arrest. However, Raskolnikov's nerves continue to wear thin, and he is constantly struggling with the idea of confessing, though he knows that he can never be truly convicted. He turns to Sonya for support and confesses his crime to her. By coincidence, Svidrigaïlov has taken up residence in a room next to Sonya's and overhears the entire confession. When the two men meet face to face, Svidrigaïlov acknowledges this fact, and suggests that he may use it against him, should he need to. Svidrigaïlov also speaks of his own past, and Raskolnikov grows to suspect that the rumors about his having committed several murders are true. In a later conversation with Dounia, Svidrigaïlov denies that he had a hand in the death of his wife. Raskolnikov is at this point completely torn; he is urged by Sonya to confess, and Svidrigaïlov's testimony could potentially convict him. Furthermore, Porfiry confronts Raskolnikov with his suspicions and assures him confession would substantially lighten his sentence. Meantime, Svidrigaïlov attempts to seduce Avdotya, but when he realizes that she will never love him, he lets her go. He then spends a night in confusion and in the morning shoots himself. This same morning, Raskolnikov goes again to Sonya, who again urges him to confess and to clear his conscience. He makes his way to the police station, where he is met by the news of Svidrigaïlov's suicide. He hesitates a moment, thinking again that he might get away with a perfect crime, but is persuaded by Sonya to confess. The epilogue tells of how Raskolnikov is sentenced to penal servitude in Siberia, where Sonya follows him. Avdotya and Razumikhin marry and are left in a happy position by the end of the novel, while Pulkheria, Raskolnikov's mother, falls ill and dies, unable to cope with her son's situation. Raskolnikov himself struggles in Siberia. It is only after some time in prison that his redemption and moral regeneration begin under Sonya's loving influence.
}''\\*

\chapter{Plagát}

